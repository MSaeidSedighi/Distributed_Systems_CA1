\documentclass[a4paper,12pt]{article}
\usepackage[utf8]{inputenc}
\usepackage{graphicx}
\usepackage{listings}
\usepackage{xcolor}

% Define colors for code
\definecolor{codegreen}{rgb}{0,0.6,0}
\definecolor{codegray}{rgb}{0.5,0.5,0.5}
\definecolor{codepurple}{rgb}{0.58,0,0.82}
\definecolor{backcolour}{rgb}{0.95,0.95,0.92}

% Code listing style
\lstset{
    backgroundcolor=\color{backcolour},
    commentstyle=\color{codegreen},
    keywordstyle=\color{magenta},
    numberstyle=\tiny\color{codegray},
    stringstyle=\color{codepurple},
    basicstyle=\ttfamily\footnotesize,
    breakatwhitespace=true,
    breaklines=true,
    captionpos=b,
    keepspaces=true,
    numbers=left,
    numbersep=5pt,
    showspaces=false,
    showstringspaces=false,
    frame=single,
    rulecolor=\color{black},
    tabsize=4,
    language=Go
}

\title{Detailed Report on MapReduce Implementation}
\author{Mohamad Saeed sedighi - 810100179\\Mohamad Hosein Motaee - 810199493\\Mahdi}

\begin{document}

\maketitle

\section{Introduction}
The report is structured to include a detailed examination of each file's code, an explanation of its functionality, and an analysis of the test output from running the \texttt{Test} script, as depicted in the attached image.

\section{Code Analysis}

\subsection{coordinator.go}
\begin{lstlisting}
package mr

import (
	"log"
	"net"
	"net/http"
	"net/rpc"
	"os"
	"sync"
	"time"
)

type Coordinator struct {
	mu          sync.Mutex
	mapTasks    []Task
	reduceTasks []Task
	mapDone     bool
	reduceDone  bool
	nReduce     int
	nMap        int
	tasks       map[int]*TaskTracker
}

type TaskTracker struct {
	Status    TaskState
	WorkerId  string
	StartTime time.Time
}

func (c *Coordinator) server() {
	rpc.Register(c)
	rpc.HandleHTTP()
	sockname := coordinatorSock()
	os.Remove(sockname)
	l, e := net.Listen("unix", sockname)
	if e != nil {
		log.Fatal("listen error:", e)
	}
	go http.Serve(l, nil)
}

func (c *Coordinator) monitor() {
	for {
		c.mu.Lock()
		if c.Done() {
			c.mu.Unlock()
			return
		}
		
		now := time.Now()
		for _, tracker := range c.tasks {
			if tracker.Status == InProgress && now.Sub(tracker.StartTime) > 10*time.Second {
				tracker.Status = Pending
				tracker.WorkerId = ""
				tracker.StartTime = time.Time{}
			}
		}
		c.mu.Unlock()
		time.Sleep(time.Second)
	}
}

func (c *Coordinator) Done() bool {
	return c.mapDone && c.reduceDone
}

func MakeCoordinator(files []string, nReduce int) *Coordinator {
	c := Coordinator{
		mapTasks:    make([]Task, len(files)),
		reduceTasks: make([]Task, nReduce),
		tasks:       map[int]*TaskTracker{},
		nReduce:     nReduce,
		nMap:        len(files),
	}

	for i, file := range files {
		c.mapTasks[i] = Task{Map, i, file, nReduce, len(files)}
		c.tasks[i] = &TaskTracker{Pending, "", time.Time{}}
	}

	for i := 0; i < nReduce; i++ {
		c.reduceTasks[i] = Task{Reduce, i, "", nReduce, len(files)}
		c.tasks[len(files)+i] = &TaskTracker{Pending, "", time.Time{}}
	}

	go c.monitor()
	c.server()
	return &c
}

func (c *Coordinator) AssignTask(args *TaskArgs, reply *TaskReply) error {
	c.mu.Lock()
	defer c.mu.Unlock()

	if !c.mapDone {
		for i, task := range c.mapTasks {
			if c.tasks[i].Status == Pending {
				c.tasks[i].Status = InProgress
				c.tasks[i].WorkerId = args.WorkerId
				c.tasks[i].StartTime = time.Now()
				reply.Task = task
				return nil
			}
		}
		c.mapDone = true
		for i := 0; i < c.nMap; i++ {
			if c.tasks[i].Status != Completed {
				c.mapDone = false
				break
			}
		}
	}

	if c.mapDone && !c.reduceDone {
		for i, task := range c.reduceTasks {
			taskId := c.nMap + i
			if c.tasks[taskId].Status == Pending {
				c.tasks[taskId].Status = InProgress
				c.tasks[taskId].WorkerId = args.WorkerId
				c.tasks[taskId].StartTime = time.Now()
				reply.Task = task
				return nil
			}
		}
		c.reduceDone = true
		for i := c.nMap; i < c.nMap+c.nReduce; i++ {
			if c.tasks[i].Status != Completed {
				c.reduceDone = false
				break
			}
		}
	}

	reply.Task = Task{Type: None}
	return nil
}

func (c *Coordinator) TaskDone(args *ReportArgs, reply *ReportReply) error {
	c.mu.Lock()
	defer c.mu.Unlock()

	taskId := args.TaskNum
	if args.TaskType == Reduce {
		taskId += c.nMap
	}

	if c.tasks[taskId].WorkerId != args.WorkerId {
		reply.Success = false
		return nil
	}

	c.tasks[taskId].Status = Completed
	reply.Success = true
	return nil
}
\end{lstlisting}

\subsubsection{Explanation}
The \texttt{coordinator.go} file is the backbone of the MapReduce system, implementing the \texttt{Coordinator} struct and its associated methods. The coordinator is responsible for orchestrating the entire MapReduce workflow, including task assignment, progress tracking, and fault tolerance.

- \texttt{Structs}:
  - \texttt{Task}: Represents a MapReduce task, which could be a map or reduce task. Fields include:
    - \texttt{Type}: Indicates whether the task is a map or reduce task.
    - \texttt{Num}: The task number.
    - \texttt{Filename}: The input file for map tasks (empty for reduce tasks).
    - \texttt{NumReduce} and \texttt{NumMap}: The number of reduce and map tasks.
  - \texttt{TaskArgs}: Contains the worker’s ID.
  - \texttt{TaskReply}: Contains the task assigned by the coordinator.
  - \texttt{ReportArgs}: Contains the worker’s ID, task number, and task type (map or reduce).
  - \texttt{ReportReply}: Indicates whether the task completion report was successful.

- \texttt{Methods}:
  - \texttt{server()}: Initializes an RPC server over a Unix socket (generated by \texttt{coordinatorSock()}). It registers the coordinator for RPC calls, removes any existing socket file, and starts an HTTP server in a goroutine to handle worker requests.
  - \texttt{monitor()}: Runs continuously in a goroutine to detect stalled tasks. It locks the mutex, checks each task in progress, and resets tasks to \texttt{Pending} if they exceed a 10-second timeout, enhancing fault tolerance by reassigning failed tasks.
  - \texttt{Done()}: Returns \texttt{true} when both map and reduce phases are complete, allowing the system to terminate gracefully.
  - \texttt{MakeCoordinator()}: Constructs a new coordinator instance. It initializes map tasks from input files, creates reduce tasks, sets up task trackers, and launches the monitor and server goroutines.
  - \texttt{AssignTask()}: An RPC method that assigns tasks to workers. It prioritizes map tasks until all are completed (\texttt{mapDone} becomes \texttt{true}), then assigns reduce tasks. If no tasks are available, it returns a \texttt{None} task type. The method updates task status and tracks worker assignment.
  - \texttt{TaskDone()}: An RPC method that marks a task as completed. It validates the worker’s ID and updates the task status to \texttt{Completed}. If the report is valid, it returns \texttt{true}, signaling success.

\subsection{rpc.go}
\begin{lstlisting}
package mr

import (
	"net/rpc"
)

type Task struct {
	Type      TaskType
	Num       int
	Filename  string
	NumReduce int
	NumMap    int
}

type TaskArgs struct {
	WorkerId string
}

type TaskReply struct {
	Task Task
}

type ReportArgs struct {
	WorkerId string
	TaskNum  int
	TaskType TaskType
}

type ReportReply struct {
	Success bool
}

type TaskType int

const (
	Map TaskType = iota
	Reduce
	None
)

func call(rpcname string, args interface{}, reply interface{}) bool {
	client, err := rpc.DialHTTP("unix", coordinatorSock())
	if err != nil {
		return false
	}
	defer client.Close()

	err = client.Call(rpcname, args, reply)
	if err != nil {
		return false
	}
	return true
}
\end{lstlisting}

\subsubsection{Explanation}
The \texttt{rpc.go} file defines the data structures used for remote procedure calls (RPCs) between the coordinator and the workers. It also contains the \texttt{call()} function to facilitate communication with the coordinator.

- \texttt{Structs}:
  - \texttt{Task}: Represents a MapReduce task, which could be a map or reduce task. Fields include:
    - \texttt{Type}: Indicates whether the task is a map or reduce task.
    - \texttt{Num}: The task number.
    - \texttt{Filename}: The input file for map tasks (empty for reduce tasks).
    - \texttt{NumReduce} and \texttt{NumMap}: The number of reduce and map tasks.
  - \texttt{TaskArgs}: Contains the worker’s ID.
  - \texttt{TaskReply}: Contains the task assigned by the coordinator.
  - \texttt{ReportArgs}: Contains the worker’s ID, task number, and task type (map or reduce).
  - \texttt{ReportReply}: Indicates whether the task completion report was successful.
  - \texttt{TaskType}: An enumeration defining the three possible types of tasks: Map, Reduce, and None.

- \texttt{Function}:
  - \texttt{call()}: This function is used by workers to send RPC requests to the coordinator. It opens a connection to the coordinator via a Unix socket and sends the RPC call. If the call is successful, it returns \texttt{true}, otherwise \texttt{false}.

\subsection{worker.go}
\begin{lstlisting}
package mr

import (
	"fmt"
	"log"
	"os"
	"time"
)

func worker() {
	for {
		var args TaskArgs
		var reply TaskReply
		if !call("Coordinator.AssignTask", &args, &reply) {
			log.Fatal("Call to Coordinator failed!")
		}

		task := reply.Task
		if task.Type == None {
			return
		}

		if task.Type == Map {
			// Perform Map Task
			err := doMapTask(task)
			if err != nil {
				log.Printf("Map task failed: %v", err)
				continue
			}
		} else if task.Type == Reduce {
			// Perform Reduce Task
			err := doReduceTask(task)
			if err != nil {
				log.Printf("Reduce task failed: %v", err)
				continue
			}
		}

		var reportArgs ReportArgs
		var reportReply ReportReply
		reportArgs.WorkerId = args.WorkerId
		reportArgs.TaskNum = task.Num
		reportArgs.TaskType = task.Type

		if !call("Coordinator.TaskDone", &reportArgs, &reportReply) {
			log.Fatal("Call to Coordinator failed!")
		}

		if !reportReply.Success {
			log.Printf("Failed to report task completion for task %d", task.Num)
		}
	}
}
\end{lstlisting}

\subsubsection{Explanation}
The \texttt{worker.go} file implements the worker's behavior. It continuously requests tasks from the coordinator, processes them, and reports the completion back to the coordinator.

- \texttt{worker()}: The worker function enters an infinite loop where it:
  - Calls the \texttt{Coordinator.AssignTask} RPC to request a task.
  - If the task is \texttt{None}, it terminates.
  - Depending on the task type (map or reduce), it calls the respective function (\texttt{doMapTask()} or \texttt{doReduceTask()}) to perform the task.
  - After completing the task, it calls \texttt{Coordinator.TaskDone} to report the completion of the task.

\section{Test Output and Analysis}
The output generated by running the test script \texttt{Test} demonstrates the proper functioning of the MapReduce system, as each worker completes its assigned task (either map or reduce) and reports the completion to the coordinator.

\begin{figure}[h]
    \centering
    \includegraphics[width=\textwidth]{test_output.jpg}
    \caption{Test Output}
    \label{fig:test_output}
\end{figure}

\end{document}
